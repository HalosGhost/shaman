\documentclass{latex2man}

\begin{Name}{1}{strfwthr}{Sam Stuewe}{Fetch Weather Information}{strfwthr - A native C library for fetching weather from the National Weather Service}
   \Prog{strfwthr} - A native C library for fetching weather from the National Weather Service
\end{Name}

\section{Synopsis}
#include <shaman/weather.h>

int strfwthr(char * s, size\_t max, const char * location, int scale,\\
const char * format, dwml weather);

\section{Description}
\Prog{strfwthr} is a small library which links to \Cmd{libcurl}{3} and \Cmd{libxml}{3}.
By linking to these libraries and using several simple constructions, \Prog{strfwthr} manages to comply to the National Weather Service's API.
This allows for a very simple and standard way to fetch, very well-done reported weather conditions.

\Prog{strfwthr()} parses the specified format and requests the data specifed by the user for the given location.
The scale parameter determines whether or not the fetched data will conform the the English (Imperial) systems of measurement, or the Metric scales.
Once the data has been received, it then breaks down the returned data and converts the conversion specifiers in the format string to the actual data they specified.
Finally, it returns the new string (with maximum length max) with all the requested data in s.

While this utility allows you to freely request data, the National Weather Services requests that API requests not be made more than once every 2 seconds.
Please be courteous, and follow this guideline if you intend to use \Prog{strfwthr} in a loop.

\section{Format}
The following format specifiers are permitted:

\begin{verbatim}
%%    A literal percent sign
%c    Weather condition
%d    Relative Humidity
%D    Dew point
%H    Hazard Warnings
%p    Pressure
%P    Probability of precipitation
%r    Reporter Identity
%R    Reporter Coordinates
%t    Temperature
%T    Apparent Temperature
%v    Visibility
%w    Wind Speed
%W    Wind Direction
\end{verbatim}
Some other basic escape characters may be used such as newline characters (\Bs n) and literal backslashes (\Bs\Bs).

\section{Testing}
To get to know how \Prog{strfwthr()} works, you can use the reference utility \Cmd{shaman}{1}.

\section{Author}
Copyright \copyright 2012-2013 Sam Stuewe\\
License GPLv2: GPL version 2 \URL{https://www.gnu.org/licenses/gpl-2.0.html} \\
This is free software: you are free to change and redistribute it. \\
There is NO WARRANTY, to the extent permitted by law.

Feature requests, bug reports and other comments may be submitted via GitHub (\URL{https://github.com/HalosGhost/shaman.git}) or E-mail (\Email{halosghost@archlinux.info})

\section{See Also}
\Cmd{libcurl}{3}

\LatexManEnd
